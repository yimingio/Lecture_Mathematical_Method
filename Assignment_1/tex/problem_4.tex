
\chead{Question 4} 


\begin{tcolorbox}
\textbf{Question 4} 

Given $f(x)=2-x^{2} \sin (x)$
\begin{enumerate}[label=(\alph*)]
\item Verify that the Bisection method can be applied to the function $f(x)$ on $[-1,2]$.
\item Using the error formula for the Bisection method find the number of iterations needed for accuracy $0.000001$.
\item Write a program by using MATLAB/Octave to determine an approximation to the root that is accurate to at least within $0.000001$.
\item Comment on the number of iterations in (b) and (c).
\end{enumerate}
\end{tcolorbox}

\begin{solution}\ \\
\begin{enumerate}[label=(\alph*)]
	\item
Here $f(x)-2-x^2 \sin(x)$ is continous function defined on $[-1,2]$,
\begin{equation}
	\left \{
	\begin{aligned}
		&f(-1)=2-(-1)^2 \sin(-1)=2-\sin(-1)=2.8414>0\\
		&f(2)=2-2^2 \sin (2)=-1.6371<0
	\end{aligned}
	\right.
\end{equation}

The Intermediate Value Theorem implies that a number p exists in $[-1,2]$ with $f(p) = 0$. Therefore the Bisection method can be applied to the function.

\item

Here apply the error function:
\begin{equation}
	\left|p_{n}-p\right| \leq \frac{b-a}{2^{n}} \label{error_function}
\end{equation}

Here $b=2$, $a=-1$,and accuracy $1.00 \times 10^{-6}$, input to (\ref{error_function})


\begin{equation*}
	\left|p_{N}-p\right| \leq 2^{-N}(2-(-1))=3 \times 2^{-N}<10^{-6}
\end{equation*}

\begin{equation*}
\begin{aligned}
	3 \times 10^6&<2^N\\
	N&> \log_2(3 \times 10^6) \approx 21.5165\\
	N& \geq 22
\end{aligned}	
\end{equation*}

Therefore $22$ times of iterations need for accuracy $0.000001$. 

\item The .m file is attached with file



Here is the list of the Iteration (\ref{result_sheet}), the iteration stops at 22 times with accuracy $0.000001$.



% Please add the following required packages to your document preamble:
% \usepackage{booktabs}
\begin{table}[htb]
\begin{tabular}{@{}lllll@{}}
\toprule
n  & $a_n$             & $b_n$            & $p_n$             & Error                \\ \midrule
1  & -1                & 2                & 0.500000000000000 & 1.50000000000000     \\
2  & 0.500000000000000 & 2                & 1.25000000000000  & 0.750000000000000    \\
3  & 1.25000000000000  & 2                & 1.62500000000000  & 0.375000000000000    \\
4  & 1.25000000000000  & 1.62500000000000 & 1.43750000000000  & 0.187500000000000    \\
5  & 1.25000000000000  & 1.43750000000000 & 1.34375000000000  & 0.0937500000000000   \\
6  & 1.34375000000000  & 1.43750000000000 & 1.39062500000000  & 0.0468750000000000   \\
7  & 1.39062500000000  & 1.43750000000000 & 1.41406250000000  & 0.0234375000000000   \\
8  & 1.41406250000000  & 1.43750000000000 & 1.42578125000000  & 0.0117187500000000   \\
9  & 1.41406250000000  & 1.42578125000000 & 1.41992187500000  & 0.00585937500000000  \\
10 & 1.41992187500000  & 1.42578125000000 & 1.42285156250000  & 0.00292968750000000  \\
11 & 1.41992187500000  & 1.42285156250000 & 1.42138671875000  & 0.00146484375000000  \\
12 & 1.42138671875000  & 1.42285156250000 & 1.42211914062500  & 0.000732421875000000 \\
13 & 1.42138671875000  & 1.42211914062500 & 1.42175292968750  & 0.000366210937500000 \\
14 & 1.42175292968750  & 1.42211914062500 & 1.42193603515625  & 0.000183105468750000 \\
15 & 1.42193603515625  & 1.42211914062500 & 1.42202758789063  & 9.15527343750000e-05 \\
16 & 1.42202758789063  & 1.42211914062500 & 1.42207336425781  & 4.57763671875000e-05 \\
17 & 1.42207336425781  & 1.42211914062500 & 1.42209625244141  & 2.28881835937500e-05 \\
18 & 1.42207336425781  & 1.42209625244141 & 1.42208480834961  & 1.14440917968750e-05 \\
19 & 1.42207336425781  & 1.42208480834961 & 1.42207908630371  & 5.72204589843750e-06 \\
20 & 1.42207908630371  & 1.42208480834961 & 1.42208194732666  & 2.86102294921875e-06 \\
21 & 1.42208194732666  & 1.42208480834961 & 1.42208337783813  & 1.43051147460938e-06 \\
22 & 1.42208337783813  & 1.42208480834961 & 1.42208409309387  & 7.15255737304688e-07 \\ \bottomrule
\end{tabular}
\caption{Iteration List of Bisection Method} \label{result_sheet}
\end{table}

\item
Here it is found that the time of iterations is exactly the number of turn out by error function.This is because the criterion here used to test the accuracy is same as the error function used in (b)(\ref{error_function}). 


Here we find that it required much work to do to enhance the accuracy with the Bisection Method.




\end{enumerate}










\end{solution}


